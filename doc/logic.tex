\documentclass[10pt,a4]{article}
\usepackage{luacode}

%
% packages
%

\usepackage{fontspec}
\usepackage{mathpazo}
\setmainfont[Ligatures=TeX]{TeXGyrePagella}
\linespread{1.05}

\def\OPTpagesize{210mm,297mm}
\def\OPTtopmargin{1in}
\def\OPTbottommargin{1in}
\def\OPTinnermargin{0.75in}
\def\OPToutermargin{1.0in}
\def\OPTbindingoffset{0.35in}
\usepackage[papersize={\OPTpagesize},
twoside,
includehead,
top=\OPTtopmargin,
bottom=\OPTbottommargin,
inner=\OPTinnermargin,
outer=\OPToutermargin,
bindingoffset=\OPTbindingoffset]{geometry}

\usepackage{xcolor}
\def\OPTlinkcolor{0,0.45,0} % RGB components for clickable links
\definecolor{linkcolor}{rgb}{\OPTlinkcolor}
\usepackage[backref=page,
colorlinks,
citecolor=linkcolor,
linkcolor=linkcolor,
urlcolor=linkcolor,
unicode=true]{hyperref}

\usepackage{amssymb,amsmath,amsthm}
% \usepackage{mathabx}
\usepackage{mathtools}
\usepackage{bussproofs}
\newenvironment{bprooftree}{\leavevmode\hbox\bgroup}{\DisplayProof\egroup}
\usepackage{framed}


% BNF-notation
\newcommand{\production}{\vcentcolon\vcentcolon=}
\newlength{\len}
\settowidth{\len}{\( \production \)}
\newcommand{\Mid}{\makebox[\len]{\( \vert \)}}

\newlength{\lenS}
\settowidth{\lenS}{\( = \)}
\newcommand{\MidS}{\makebox[\lenS]{\( \vert \)}}

% operator
\newcommand{\abs}[2]{\lambda #1.\,#2}
\newcommand{\Abs}[2]{\underbar{\lambda} #1.\,#2}

\newcommand{\absP}[2]{\zeta #1.\,#2}
\newcommand{\AbsP}[2]{\underbar{\zeta} #1.\,#2}

\newcommand{\app}[2]{#1 \lhd #2}
\newcommand{\App}[2]{#1 \mathbin{\underline{\lhd}} #2}

\newcommand{\appE}[2]{#1 \mathbin{@} #2}
\newcommand{\AppE}[2]{#1 \mathbin{\underline{@}} #2}

\newcommand{\appT}[2]{#1 \LHD #2}
\newcommand{\AppT}[2]{#1 \mathbin{\underline{\LHD}} #2}
% \newcommand{\AppT}[2]{#1 \UNLHD #2}

\newcommand{\appsT}[3]{#2 \diamond^{#1} #3}
\newcommand{\appi}[2]{#1 @_{i} #2}
\newcommand{\coabs}[2]{\mu #1.\,#2}
\newcommand{\coapp}[2]{#1 \diamond #2}
\newcommand{\pair}[2]{(#1, #2)}
\newcommand{\simnot}{\mathord{\sim}}
\newcommand{\LET}[3]{\mathsf{let}\ #1 = #2\ \mathsf{in}\ #3}
\newcommand{\slet}[3]{\mathsf{let}\ #1 = #2\ \mathsf{in}\ #3}

% \newcommand{\xabs}[2]{\Lambda #1.\,#2}
\newcommand{\xabs}[2]{\thunkS{(\abs{#1}{#2})}}
\newcommand{\xAbs}[2]{\underbar{\Lambda} #1.\,#2}

\newcommand{\quit}[3]{\mathsf{quit}^{#1}(#2, #3)}

% notation
% \newcommand{\subst}[2]{\{ #1 := #2 \}}
\newcommand{\subst}[2]{[#2/#1]}

\newcommand{\downT}[1]{\mathord{\downarrow}#1}
\newcommand{\upT}[1]{\mathord{\uparrow}#1}

\newcommand{\DownT}[1]{\mathord{\Downarrow}#1}
\newcommand{\UpT}[1]{\mathord{\Uparrow}#1}

\newcommand{\ir}[1]{\mathsf{#1}}

% \newcommand{\shift}[2]{\mathcal{S}#1.\,#2}
\newcommand{\shiftT}[2]{\mathcal{S}#1.\,#2}
\newcommand{\delimit}[1]{\langle #1 \rangle}

\newcommand{\dprod}[2]{\Pi (#1).\,#2}
\newcommand{\dsum}[2]{\Sigma (#1).\,#2}

\newcommand{\Dprod}[2]{\hat{\Pi} (#1).\,#2}
\newcommand{\Dsum}[2]{\widehat{\Sigma} (#1).\,#2}

\newcommand{\thunkS}[1]{\mathsf{thunk}\,#1}
\newcommand{\unthunkS}[1]{\mathsf{unthunk}\,#1}
\newcommand{\forceS}[1]{\mathsf{force}\,#1}

\newcommand{\quoteS}[2]{\mathsf{quote}(#1, #2)}
\newcommand{\unquoteS}[2]{\mathsf{unquote}\ #1\ \mathsf{in}\ #2}

\newcommand{\runS}[1]{\mathsf{run}\,#1}

\newcommand{\bind}[3]{#1 \rhd_{#2} #3}
\newcommand{\cobind}[3]{#1 \RHD_{#2} #3}

\newcommand{\vcode}[1]{\mathsf{code}\,#1}
\newcommand{\vdata}[1]{\mathsf{data}\,#1}
\newcommand{\mybox}[1]{\mathsf{box}\,#1}
\newcommand{\myunbox}[1]{\mathsf{unbox}\,#1}
\newcommand{\vcocode}[1]{\mathsf{cocode}\,#1}
\newcommand{\vcodata}[1]{\mathsf{codata}\,#1}

\newcommand{\vret}[1]{\mathsf{return}\,#1}
\newcommand{\vcoret}[1]{\mathsf{coreturn}\,#1}

\newcommand{\flip}[1]{#1^{\circlearrowleft}}
\newcommand{\flop}[1]{#1^{\CIRCLE}}

% \newcommand{\pair}[2]{(#1, #2)}

\newcommand{\toneg}[1]{#1^{-}}
\newcommand{\topos}[1]{#1^{+}}

\newcommand{\stringT}{\mathsf{string}}
\newcommand{\print}[1]{\mathsf{print}\,#1}

\newcommand{\rec}[2]{\mathsf{rec}\,#1.\,#2}
\newcommand{\corec}[2]{\mathsf{corec}\,#1\,#2}

\newcommand{\myzero}{\mathsf{zero}}
\newcommand{\mysucc}[1]{\mathsf{succ}\,#1}
\newcommand{\nat}{\mathbb{N}}

\newcommand{\FA}[2]{\forall #1.\,#2}
\newcommand{\EX}[2]{\exists #1.\,#2}

\newcommand{\deref}[1]{\mathsf{deref}\,#1}
\newcommand{\mydia}[1]{\mathsf{dia}\,#1}
\newcommand{\coderef}[1]{\mathsf{coderef}\,#1}
\newcommand{\cothunk}[1]{\mathsf{cothunk}\,#1}
\newcommand{\coforce}[1]{\mathsf{coforce}\,#1}

\newcommand{\myarray}[2]{[#2\,; #1]}

\newcommand{\sep}{\,;\ }

\newcommand{\ctx}[1]{\Vdash #1}
% \newcommand{\ctx}[1]{#1\,\mathsf{ctx}}

\newcommand{\luniv}{\mathbb{L}}
% \newcommand{\univ}{\Box \luniv}
\newcommand{\univ}{\mathbb{U}}

\newcommand{\affine}[1]{\( #1 \) is affine}
\newcommand{\relevant}[1]{\( #1 \) is relevant}

\newcommand{\nuabs}[2]{\nu #1.\,#2}
\newcommand{\conuabs}[2]{\overline{\nu} #1.\,#2}
\newcommand{\tensor}[2]{\mathsf{tensor}(#1, #2)}
\newcommand{\tpar}[2]{\mathsf{dispatch}(#1, #2)}
\newcommand{\tleft}[1]{\mathsf{left}\,#1}
\newcommand{\tright}[1]{\mathsf{right}\,#1}
\newcommand{\tlet}[4]{\mathsf{let}\ \tensor{#1}{#2} = #3\ \mathsf{in}\ #4}
\newcommand{\E}{\mathbb{E}}
\newcommand{\J}{\mathcal{J}}

\usepackage{cmll}



\usepackage{tikz}
\usetikzlibrary{cd}

\usepackage{amssymb,amsmath,amsthm,stmaryrd,mathrsfs,wasysym}
\usepackage{enumitem,mathtools,xspace}
\usepackage{xstring}

\title{}
\date{}
\author{}

\begin{document}

% \maketitle

\small

\begin{framed}
  \begin{center}
    \begin{bprooftree}
      \AxiomC{\( \vphantom{\Gamma} \)}
      \RightLabel{\( \ir{(empty)} \)}
      \UnaryInfC{\( \ctx{\emptyset} \)}
    \end{bprooftree}
    \begin{bprooftree}
      \AxiomC{\( \ctx{\Box \Gamma} \)}
      \RightLabel{\( \ir{(univ)} \)}
      \UnaryInfC{\( \Box \Gamma \vdash \luniv_{i} : \Box \luniv_{i + 1} \)}
    \end{bprooftree}
    \begin{bprooftree}
      \AxiomC{\( \Gamma \vdash A : \Box \luniv_{i} \)}
      \RightLabel{\( \ir{(hier)} \)}
      \UnaryInfC{\( \Gamma \vdash A : \Box \luniv_{i + 1} \)}
    \end{bprooftree}
    \\
    \vspace{1.5em}
    \begin{bprooftree}
      \AxiomC{\( \Gamma \vdash A : \Box \luniv_{i} \)}
      \RightLabel{\( \ir{(ext)} \)}
      \UnaryInfC{\( \ctx{\Gamma, x : A} \)}
    \end{bprooftree}
    \begin{bprooftree}
      \AxiomC{\( \ctx{\Box \Gamma, x : A} \)}
      \RightLabel{\( \ir{(var)} \)}
      \UnaryInfC{\( \Box \Gamma, x : A \vdash x : A \)}
    \end{bprooftree}
    \\
    \vspace{1.5em}
    \begin{bprooftree}
      \AxiomC{\( \Gamma \vdash A : \Box \luniv_{i} \)}
      \AxiomC{\( \Delta, x : A \vdash B : \Box \luniv_{i} \)}
      \RightLabel{\( (\Pi) \)}
      \BinaryInfC{\( \Gamma, \Delta \vdash \dprod{x : A}{B} : \Box \luniv_{i} \)}
    \end{bprooftree}
    \\
    \vspace{1.5em}
    \begin{bprooftree}
      \AxiomC{\( \Gamma, x : A \vdash e : B \)}
      \RightLabel{\( (\Pi_{\ir{i}}) \)}
      \UnaryInfC{\( \Gamma \vdash \abs{x}{e} : \dprod{x : A}{B} \)}
    \end{bprooftree}
    \begin{bprooftree}
      \AxiomC{\( \Gamma \vdash e_{1} : \dprod{x : A}{B}\)}
      \AxiomC{\( \Delta \vdash e_{2} : A \)}
      \RightLabel{\( (\Pi_{\ir{e}}) \)}
      \BinaryInfC{\( \Gamma, \Delta \vdash \appE{e_{1}}{e_{2}} : B \subst{x}{e_{1}} \)}
    \end{bprooftree}
    \\
    \vspace{1.5em}
    \begin{bprooftree}
      \AxiomC{\( \Gamma \vdash A : \Box \luniv_{i} \)}
      \AxiomC{\( \Delta, x : A \vdash B : \Box \luniv_{i} \)}
      \RightLabel{\( (\Sigma) \)}
      \BinaryInfC{\( \Gamma, \Delta \vdash \dsum{x : A}{B} : \Box \luniv_{i} \)}
    \end{bprooftree}
    \\
    \vspace{1.5em}
    \begin{bprooftree}
      \AxiomC{\( \Gamma \vdash e_{1} : A \)}
      \AxiomC{\( \Delta \vdash e_{2} : B \subst{x}{e_{1}} \)}
      \RightLabel{\( (\Sigma_{\ir{i}}) \)}
      \BinaryInfC{\( \Gamma, \Delta \vdash \pair{e_{1}}{e_{2}} : \dsum{x : A}{B} \)}
    \end{bprooftree}
    \begin{bprooftree}
      \AxiomC{\( \Gamma \vdash e_{1} : \dsum{x : A}{B} \)}
      \AxiomC{\( \Delta, x : A, y : B \vdash e_{2} : C \subst{z}{\pair{x}{y}} \)}
      \RightLabel{\( (\Sigma_{\ir{e}}) \)}
      \BinaryInfC{\( \Gamma, \Delta \vdash \LET{\pair{x}{y}}{e_{1}}{e_{2}} : C \subst{z}{e_{1}} \)}
    \end{bprooftree}
    \\
    \vspace{1.5em}
    \begin{bprooftree}
      \AxiomC{\( \Gamma \vdash A : \Box \luniv_{i} \)}
      \RightLabel{\( (\Box) \)}
      \UnaryInfC{\( \Gamma \vdash \Box A : \Box \luniv_{i} \)}
    \end{bprooftree}
    \\
    \vspace{1.5em}
    \begin{bprooftree}
      \AxiomC{\( \Box \Gamma \vdash e : A \)}
      \RightLabel{\( (\Box_{\ir{i}}) \)}
      \UnaryInfC{\( \Box \Gamma \vdash \mybox{e} : \Box A \)}
    \end{bprooftree}
    \begin{bprooftree}
      \AxiomC{\( \Box \Gamma \vdash e : \Box A \)}
      \RightLabel{\( (\Box_{\ir{e}}) \)}
      \UnaryInfC{\( \Box \Gamma \vdash \myunbox{e} : A \)}
    \end{bprooftree}
    % \\
    % \vspace{1.5em}
    % \begin{bprooftree}
    %   \AxiomC{\( \ctx{\Gamma} \)}
    %   \RightLabel{\( (\top) \)}
    %   \UnaryInfC{\( \Gamma \vdash \top : \Box \luniv_{i} \)}
    % \end{bprooftree}
    % \begin{bprooftree}
    %   \AxiomC{\( \vphantom{\Gamma} \)}
    %   \RightLabel{\( (\top_{\ir{i}}) \)}
    %   \UnaryInfC{\( \vdash \ir{unit} : \top \)}
    % \end{bprooftree}
    \\
    \vspace{1.5em}
    \begin{bprooftree}
      \AxiomC{\( \Box \Gamma, x : \Box A \vdash e : \Box A \)}
      \RightLabel{\( \ir{(rec)} \)}
      \UnaryInfC{\( \Box \Gamma \vdash \rec{x}{e} : \Box A \)}
    \end{bprooftree}
    \\
    \vspace{1.5em}
    \begin{bprooftree}
      \AxiomC{\( \Gamma \vdash e : B \)}
      \RightLabel{\( \ir{(wkg)} \)}
      \UnaryInfC{\( \Gamma, x : \Box A \vdash e : B \)}
    \end{bprooftree}
    \begin{bprooftree}
      \AxiomC{\( \Gamma, x : \Box A, y : \Box A \vdash e : B \)}
      \RightLabel{\( \ir{(ctr)} \)}
      \UnaryInfC{\( \Gamma, x : \Box A \vdash e \subst{y}{x} : B \)}
    \end{bprooftree}
  \end{center}
\end{framed}

\normalsize

A type \( A \) is relevant if \( \vdash e : A \to A \otimes A \) is derivable for some \( e \).

A type \( A \) is affine if \( \vdash e : A \to \top \) is derivable for some \( e \).

A term \( e \) is a well-typed closed proper term if (1) \( \vdash e : A \) is derivable for some type \( A \), and (2) for any variable \( x \) in \( e \), if it occurs more than once, then the type of \( x \) is relevant. If it doesn't occur, then the type of \( x \) is affine. (3) All the free variables in a recursion-term (\( \rec{x}{e} \)) must be relevant.

% \( \Box \luniv_{i} \) is a shorthand of \( \Box \luniv_{i} \).

\bibliographystyle{plain}
\bibliography{library}

\end{document}
